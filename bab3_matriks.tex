\documentclass{beamer}
\usepackage[utf8]{inputenc}
\usepackage[bahasai]{babel}
\usepackage{amsmath}
\title{Matriks dan Determinan}
\author{Bagus Tris Atmaja\\Institut Teknologi Sepuluh Nopember}
\date{\today}
\begin{document}
	\frame{\titlepage}
	\frame{
		\frametitle{Table of Contents}
		\tableofcontents
	}
	
\section{Definisi-definisi matriks}
\begin{frame}[t,fragile]{Definisi}
Matriks adalah suatu susunan deretan empat persegi panjang dari bilangan-bilangan yang tersusun dalam $m$ baris dan $n$ kolom dalam tanda kurung.\\
Contoh:\\

$$A = \begin{bmatrix}
  a_{11} & a_{12} & a_{13} & a_{14}\\
  a_{21} & a_{22} & a_{23} & a_{24}\\
  a_{31} & a_{32} & a_{33} & a_{34}
 \end{bmatrix} $$

adalah suatu matriks berukuran 3x4. Ditulis $A_{3x4}$ (3 baris 4 kolom).\\
Matriks bujur sangkar adalah suatu matriks yang jumlah barisnya sama dengan jumlah kolomnya (disebut matriks dengan orde $n$).\\
Matriks Simetrik adalah matrik bujur sangkar jika ditransposekan hasilnya sama.\\
$$A = A^T$$
\end{frame}

\begin{frame}[t,fragile]{Matriks}
$$A = \begin{bmatrix}
    2 & 1 & 3 \\
    1 & 5 & 0 \\
    3 & 0 & 1
   \end{bmatrix} $$

\begin{itemize}
\item \underline{Trace}: Jumlah elemen diagonal pokok suatu matriks bujur sangkar.\\
$$Trace~~A = 2 + 5 + 1 = 8$$
\item \underline{Matriks singular}: Jika determinan matriks bujur sangkar adalah nol.
\item \underline{Matriks baris}: Jika $m=1$\\
$$ B = \bigl[ b_1 ~~~ b_2 ~~~ b_3 \bigr]$$ 
\end{itemize}
\end{frame}

\begin{frame}[t,fragile]{Matriks}
\begin{itemize}
\item \underline{Matriks kolom}: Jika $n=1$ \\
$$ C = \begin{bmatrix}
       	C_1 \\
	    C_2 \\
	    C_3 
        \end{bmatrix} $$
\item \underline{Matris nol}: matrik yang semua elemennya adalah nol.\\
$$ B = \begin{bmatrix}
       	0 & 0 & 0\\
	    0 & 0 & 0 
        \end{bmatrix} $$
\item \underline{Matriks Satuan}:\\
$$ C = \begin{bmatrix}
       	1 & 0 & 0 \\
	    0 & 1 & 0 \\
	    0 & 0 & 1
        \end{bmatrix} $$
\end{itemize}
\end{frame}

\begin{frame}[t,fragile]{Matriks}
\begin{itemize}
\item \underline{Matriks Diagonal}: Matriks bujur sangkar yang mempunyai nilai pada elemen diagonal sedang elemen lainnya nol.
$$ a_{ii} = \begin{bmatrix}
       	a_{11} & 0 & 0 \\
	    0 & a_{12} & 0 \\
	    0 & 0 & a_{13} 
        \end{bmatrix} $$
\item \underline{Transpose}: transpose dari matriks A adalah matriks dimana baris dan kolom dipertukarkan.
\begin{columns}
\column{.5\textwidth}
$$ A = \begin{bmatrix}
       	a_{11} & a_{12} & a_{13}\\
	    a_{21} & a_{22} & a_{23}\\
        \end{bmatrix} $$
\column{.5\textwidth}
$$ \Longrightarrow A^T = \begin{bmatrix}
       	a_{11} & a_{21}\\
	    a_{12} & a_{22}\\
        a_{13} & a_{23} 
        \end{bmatrix} $$
\end{columns}
\item \underline{Minor}: Minor $M_{ij}$ dari suatu matriks A dibentuk dengan menghilangkan baris ke $i$ dan kolom ke $j$ dari matriks asal.
\begin{columns}
\column{.5\textwidth}
$$ A = \begin{bmatrix}
       	a_{11} & a_{12} & a_{13}\\
	    a_{21} & a_{22} & a_{23}\\
        a_{31} & a_{32} & a_{33} 
        \end{bmatrix} $$
\column{.5\textwidth}
$$ \Longrightarrow M_{12} = \begin{bmatrix}
       	a_{21} & a_{21}\\
        a_{31} & a_{33} 
        \end{bmatrix} $$
\end{columns}
\end{itemize}
\end{frame}

\begin{frame}[t,fragile]{Matriks}
\begin{itemize}
\item \underline{Kofaktor}: Kofaktor $C_{ij}$ adalah sama dengan minor yang tertanda $(-1)^{i+j} M_{ij}$.\\
    $$ C_{12} = (-1)^{1+2} ~ M_{12} = - M_{12} $$
\item \underline{Matriks Inversi}: Inversi $A^{(-1)}$ dari matriks $A$ memenuhi hubungan,\\ 
    $$ A^{-1} A = A A^{-1} = I $$
\item \underline{Matriks orthogonal}: Matriks yang mememenuhi hubungan,\\
    $$ A^T A = A A^T = I$$ 
\end{itemize}
\end{frame}

\begin{frame}[t,fragile]{Adjoint Matriks}
Suatu matriks adjoint dari matriks bujur sangkar $A$ adalah transpose matriks kofaktor $A$. \\
Misalkan matriks kofaktor $A$ adalah:
$$ C_{ij} = \begin{bmatrix}
            C_{11} & C_{12} & C_{13} \\
            C_{21} & C_{22} & C_{23} \\
            C_{31} & C_{32} & C_{33}
            \end{bmatrix} $$
$$ Adj~A = \bigl[ C_{ij}^T \bigr] = \bigl[ C_{ji} \bigr] =  \begin{bmatrix}
                                                            C_{11} & C_{21} & C_{31} \\
                                                            C_{12} & C_{22} & C_{32} \\
                                                            C_{13} & C_{23} & C_{33}
                                                            \end{bmatrix} $$
\end{frame}

\section{Aturan Operasi Matriks}	
\begin{frame}[t,fragile]{Penjumlahan dan Pengurangan Matriks}
Dua matriks $A$ dan $B$ dapat dijumlahkan atau dikurangkan jika dan hanya jika kedua matriks tersebut mempunyai ukuran yang sama.\\
\vspace{0.3cm}
Jika $A = [a_{ij}]$ dan $B = [b_{ij}]$ maka:
\begin{itemize}
\item Penjumlahan: $(A + B) = [a_{ij} + b_{ij}]$
\item Pengurangan: $(A - B) = [a_{ij} - b_{ij}]$
\end{itemize}
\vspace{0.3cm}
\textbf{Contoh:}\\
$$ A = \begin{bmatrix}
       	2 & 3 \\
	    1 & 4
        \end{bmatrix}, ~~ B = \begin{bmatrix}
       	5 & 1 \\
	    2 & 3
        \end{bmatrix} $$
$$ A + B = \begin{bmatrix}
       	7 & 4 \\
	    3 & 7
        \end{bmatrix}, ~~ A - B = \begin{bmatrix}
       	-3 & 2 \\
	    -1 & 1
        \end{bmatrix} $$
\end{frame}

\begin{frame}[t,fragile]{Perkalian Skalar dengan Matriks}
Jika $k$ adalah suatu bilangan skalar dan $A = [a_{ij}]$ adalah matriks, maka perkalian skalar dengan matriks adalah:
$$ kA = k[a_{ij}] = [ka_{ij}] $$
\vspace{0.3cm}
\textbf{Contoh:}\\
Jika $k = 3$ dan 
$$ A = \begin{bmatrix}
       	2 & -1 & 3 \\
	    0 & 4 & 1
        \end{bmatrix} $$
Maka:
$$ 3A = \begin{bmatrix}
       	6 & -3 & 9 \\
	    0 & 12 & 3
        \end{bmatrix} $$
\end{frame}

\begin{frame}[t,fragile]{Perkalian Matriks}
Matriks $A$ berukuran $m \times n$ dapat dikalikan dengan matriks $B$ berukuran $n \times p$ menghasilkan matriks $C$ berukuran $m \times p$.\\
\vspace{0.2cm}
Syarat: Jumlah kolom matriks pertama harus sama dengan jumlah baris matriks kedua.\\
\vspace{0.2cm}
Elemen $c_{ij}$ dari matriks hasil $C$ adalah:
$$ c_{ij} = \sum_{k=1}^{n} a_{ik} \cdot b_{kj} $$
\vspace{0.2cm}
\textbf{Contoh:}\\
$$ A_{2 \times 3} = \begin{bmatrix}
       	1 & 2 & 3 \\
	    4 & 5 & 6
        \end{bmatrix}, ~~ B_{3 \times 2} = \begin{bmatrix}
       	1 & 2 \\
	    0 & 1 \\
        1 & 0
        \end{bmatrix} $$
\end{frame}

\begin{frame}[t,fragile]{Perkalian Matriks (lanjutan)}
$$ A_{2 \times 3} = \begin{bmatrix}
       	1 & 2 & 3 \\
	    4 & 5 & 6
        \end{bmatrix}, ~~ B_{3 \times 2} = \begin{bmatrix}
       	1 & 2 \\
	    0 & 1 \\
        1 & 0
        \end{bmatrix} $$
\vspace{0.3cm}
$$ C_{2 \times 2} = AB = \begin{bmatrix}
       	(1 \cdot 1 + 2 \cdot 0 + 3 \cdot 1) & (1 \cdot 2 + 2 \cdot 1 + 3 \cdot 0) \\
	    (4 \cdot 1 + 5 \cdot 0 + 6 \cdot 1) & (4 \cdot 2 + 5 \cdot 1 + 6 \cdot 0)
        \end{bmatrix} $$
\vspace{0.3cm}
$$ C = \begin{bmatrix}
       	4 & 4 \\
	    10 & 13
        \end{bmatrix} $$
\vspace{0.2cm}
\textbf{Catatan:} Perkalian matriks tidak komutatif, artinya $AB \neq BA$ (secara umum).
\end{frame}

\begin{frame}[t,fragile]{Sifat-sifat Operasi Matriks}
Misalkan $A$, $B$, dan $C$ adalah matriks-matriks dengan ukuran yang sesuai, dan $k$, $m$ adalah skalar, maka berlaku:
\begin{enumerate}
\item $A + B = B + A$ (komutatif penjumlahan)
\item $(A + B) + C = A + (B + C)$ (asosiatif penjumlahan)
\item $A + O = O + A = A$ (identitas penjumlahan)
\item $A + (-A) = O$ (invers penjumlahan)
\item $k(A + B) = kA + kB$ (distributif kiri)
\item $(k + m)A = kA + mA$ (distributif kanan)
\item $(km)A = k(mA)$ (asosiatif perkalian skalar)
\item $(AB)C = A(BC)$ (asosiatif perkalian matriks)
\item $A(B + C) = AB + AC$ (distributif kiri perkalian)
\item $(A + B)C = AC + BC$ (distributif kanan perkalian)
\item $AI = IA = A$ (identitas perkalian)
\end{enumerate}
\end{frame}

\section{Determinan Tingkat N}
\begin{frame}[t,fragile]{Definisi Determinan}
\textbf{Determinan} adalah suatu bilangan yang dihubungkan dengan suatu matriks bujur sangkar.\\
\vspace{0.3cm}
Untuk matriks $A$ berukuran $n \times n$, determinannya ditulis sebagai:
$$ \det(A) \quad \text{atau} \quad |A| $$
\vspace{0.3cm}
\textbf{Sifat penting:}
\begin{itemize}
\item Determinan hanya didefinisikan untuk matriks bujur sangkar
\item Determinan menghasilkan suatu bilangan skalar
\item Jika $\det(A) = 0$, matriks $A$ disebut \textbf{singular} (tidak memiliki invers)
\item Jika $\det(A) \neq 0$, matriks $A$ disebut \textbf{non-singular} (memiliki invers)
\end{itemize}
\end{frame}

\section{Ekspansi Laplace}
\begin{frame}[t,fragile]{Ekspansi Laplace}
\textbf{Ekspansi Laplace} adalah metode untuk menghitung determinan matriks dengan mengekspansikan sepanjang suatu baris atau kolom.\\
\vspace{0.3cm}
Determinan matriks $A$ berukuran $n \times n$ dapat dihitung dengan ekspansi baris ke-$i$:
$$ \det(A) = \sum_{j=1}^{n} a_{ij} C_{ij} = \sum_{j=1}^{n} a_{ij} (-1)^{i+j} M_{ij} $$
atau ekspansi kolom ke-$j$:
$$ \det(A) = \sum_{i=1}^{n} a_{ij} C_{ij} = \sum_{i=1}^{n} a_{ij} (-1)^{i+j} M_{ij} $$
dimana:
\begin{itemize}
\item $a_{ij}$ adalah elemen matriks pada baris $i$ kolom $j$
\item $C_{ij}$ adalah kofaktor
\item $M_{ij}$ adalah minor (determinan submatriks)
\end{itemize}
\end{frame}

\begin{frame}[t,fragile]{Determinan Orde 1}
Untuk matriks $1 \times 1$:
$$ A = [a_{11}] $$
Determinannya adalah:
$$ \det(A) = a_{11} $$
\vspace{0.3cm}
\textbf{Contoh:}
$$ A = [5] \quad \Longrightarrow \quad \det(A) = 5 $$
\end{frame}

\begin{frame}[t,fragile]{Determinan Matriks 2x2}
Untuk matriks $2 \times 2$:
$$ A = \begin{bmatrix}
       	a_{11} & a_{12} \\
	    a_{21} & a_{22}
        \end{bmatrix} $$
Determinannya adalah:
$$ \det(A) = |A| = a_{11}a_{22} - a_{12}a_{21} $$
\vspace{0.3cm}
\textbf{Contoh:}\\
$$ A = \begin{bmatrix}
       	3 & 2 \\
	    1 & 4
        \end{bmatrix} $$
$$ \det(A) = (3)(4) - (2)(1) = 12 - 2 = 10 $$
\end{frame}

\begin{frame}[t,fragile]{Determinan Matriks 3x3 dengan Ekspansi Laplace}
Untuk matriks $3 \times 3$, kita ekspansi sepanjang baris pertama:
$$ A = \begin{bmatrix}
       	a_{11} & a_{12} & a_{13} \\
	    a_{21} & a_{22} & a_{23} \\
	    a_{31} & a_{32} & a_{33}
        \end{bmatrix} $$
$$ \det(A) = a_{11}C_{11} + a_{12}C_{12} + a_{13}C_{13} $$
dimana:
\begin{align*}
C_{11} &= (+1) \begin{vmatrix} a_{22} & a_{23} \\ a_{32} & a_{33} \end{vmatrix} = a_{22}a_{33} - a_{23}a_{32} \\
C_{12} &= (-1) \begin{vmatrix} a_{21} & a_{23} \\ a_{31} & a_{33} \end{vmatrix} = -(a_{21}a_{33} - a_{23}a_{31}) \\
C_{13} &= (+1) \begin{vmatrix} a_{21} & a_{22} \\ a_{31} & a_{32} \end{vmatrix} = a_{21}a_{32} - a_{22}a_{31}
\end{align*}
\end{frame}

\begin{frame}[t,fragile]{Contoh Ekspansi Laplace untuk Matriks 3x3}
Hitung determinan matriks berikut dengan ekspansi baris pertama:
$$ A = \begin{bmatrix}
       	2 & 1 & 3 \\
	    0 & 4 & 1 \\
	    1 & 2 & 5
        \end{bmatrix} $$
\textbf{Solusi:}
\begin{align*}
\det(A) &= 2 \begin{vmatrix} 4 & 1 \\ 2 & 5 \end{vmatrix} - 1 \begin{vmatrix} 0 & 1 \\ 1 & 5 \end{vmatrix} + 3 \begin{vmatrix} 0 & 4 \\ 1 & 2 \end{vmatrix} \\
&= 2(4 \cdot 5 - 1 \cdot 2) - 1(0 \cdot 5 - 1 \cdot 1) + 3(0 \cdot 2 - 4 \cdot 1) \\
&= 2(20 - 2) - 1(0 - 1) + 3(0 - 4) \\
&= 2(18) - 1(-1) + 3(-4) \\
&= 36 + 1 - 12 = 25
\end{align*}
\end{frame}

\begin{frame}[t,fragile]{Strategi Pemilihan Baris/Kolom}
\textbf{Tips untuk Ekspansi Laplace:}
\begin{enumerate}
\item Pilih baris atau kolom yang memiliki paling banyak elemen nol untuk mempermudah perhitungan
\item Elemen nol tidak perlu dihitung kofaktornya (karena dikalikan nol)
\item Ekspansi dapat dilakukan di baris atau kolom manapun, hasilnya sama
\end{enumerate}
\vspace{0.3cm}
\textbf{Contoh:} Untuk matriks
$$ B = \begin{bmatrix}
       	2 & 0 & 3 \\
	    0 & 0 & 1 \\
	    1 & 0 & 5
        \end{bmatrix} $$
Lebih efisien ekspansi sepanjang kolom kedua (3 elemen nol):
$$ \det(B) = 0 - 0 + 0 = 0 $$
Jadi matriks $B$ adalah matriks singular.
\end{frame}

\section{Aturan Sarrus}
\begin{frame}[t,fragile]{Aturan Sarrus}
\textbf{Aturan Sarrus} adalah metode khusus untuk menghitung determinan matriks $3 \times 3$ secara cepat.\\
\vspace{0.3cm}
Untuk matriks:
$$ A = \begin{bmatrix}
       	a_{11} & a_{12} & a_{13} \\
	    a_{21} & a_{22} & a_{23} \\
	    a_{31} & a_{32} & a_{33}
        \end{bmatrix} $$
\textbf{Langkah-langkah:}
\begin{enumerate}
\item Tulis ulang 2 kolom pertama di sebelah kanan matriks
\item Kalikan elemen-elemen diagonal dari kiri atas ke kanan bawah (positif)
\item Kalikan elemen-elemen diagonal dari kiri bawah ke kanan atas (negatif)
\item Jumlahkan hasil perkalian dengan tanda yang sesuai
\end{enumerate}
\end{frame}

\begin{frame}[t,fragile]{Aturan Sarrus - Formula}
$$ A = \begin{bmatrix}
       	a_{11} & a_{12} & a_{13} \\
	    a_{21} & a_{22} & a_{23} \\
	    a_{31} & a_{32} & a_{33}
        \end{bmatrix} 
        \begin{matrix}
       	a_{11} & a_{12} \\
	    a_{21} & a_{22} \\
	    a_{31} & a_{32}
        \end{matrix} $$
\vspace{0.3cm}
\textbf{Diagonal positif} (kiri atas ke kanan bawah):
$$ +a_{11}a_{22}a_{33} + a_{12}a_{23}a_{31} + a_{13}a_{21}a_{32} $$
\textbf{Diagonal negatif} (kiri bawah ke kanan atas):
$$ -a_{13}a_{22}a_{31} - a_{11}a_{23}a_{32} - a_{12}a_{21}a_{33} $$
\vspace{0.2cm}
$$ \det(A) = a_{11}a_{22}a_{33} + a_{12}a_{23}a_{31} + a_{13}a_{21}a_{32} $$
$$ - a_{13}a_{22}a_{31} - a_{11}a_{23}a_{32} - a_{12}a_{21}a_{33} $$
\end{frame}

\begin{frame}[t,fragile]{Contoh Aturan Sarrus}
Hitung determinan matriks berikut menggunakan Aturan Sarrus:
$$ A = \begin{bmatrix}
       	2 & 1 & 3 \\
	    0 & 4 & 1 \\
	    1 & 2 & 5
        \end{bmatrix} 
        \begin{matrix}
       	2 & 1 \\
	    0 & 4 \\
	    1 & 2
        \end{matrix} $$
\textbf{Diagonal positif:}
$$ (2)(4)(5) + (1)(1)(1) + (3)(0)(2) = 40 + 1 + 0 = 41 $$
\textbf{Diagonal negatif:}
$$ (3)(4)(1) + (2)(1)(2) + (1)(0)(5) = 12 + 4 + 0 = 16 $$
\textbf{Hasil:}
$$ \det(A) = 41 - 16 = 25 $$
\end{frame}

\begin{frame}[t,fragile]{Catatan Penting Aturan Sarrus}
\begin{itemize}
\item Aturan Sarrus \textbf{hanya berlaku} untuk matriks $3 \times 3$
\item Untuk matriks $2 \times 2$, gunakan rumus langsung
\item Untuk matriks $4 \times 4$ atau lebih besar, gunakan Ekspansi Laplace
\item Aturan Sarrus lebih cepat daripada Ekspansi Laplace untuk matriks $3 \times 3$
\item Hasil determinan sama dengan metode Ekspansi Laplace
\end{itemize}
\end{frame}

\section{Determinan Transpose}
\begin{frame}[t,fragile]{Determinan Transpose}
\textbf{Teorema:} Determinan suatu matriks sama dengan determinan transposenya.
$$ \det(A) = \det(A^T) $$
\vspace{0.3cm}
\textbf{Implikasi:}
\begin{itemize}
\item Semua sifat determinan yang berlaku untuk baris juga berlaku untuk kolom
\item Ekspansi Laplace dapat dilakukan sepanjang baris atau kolom
\item Operasi baris dan operasi kolom memiliki efek yang sama terhadap determinan
\end{itemize}
\end{frame}

\begin{frame}[t,fragile]{Bukti untuk Matriks 2x2}
Misalkan:
$$ A = \begin{bmatrix}
       	a & b \\
	    c & d
        \end{bmatrix}, \quad A^T = \begin{bmatrix}
       	a & c \\
	    b & d
        \end{bmatrix} $$
Maka:
\begin{align*}
\det(A) &= ad - bc \\
\det(A^T) &= ad - cb = ad - bc
\end{align*}
$$ \therefore \det(A) = \det(A^T) $$
\end{frame}

\begin{frame}[t,fragile]{Contoh Determinan Transpose}
Verifikasi bahwa $\det(A) = \det(A^T)$ untuk matriks berikut:
$$ A = \begin{bmatrix}
       	1 & 2 & 3 \\
	    0 & 4 & 5 \\
	    1 & 0 & 6
        \end{bmatrix}, \quad A^T = \begin{bmatrix}
       	1 & 0 & 1 \\
	    2 & 4 & 0 \\
	    3 & 5 & 6
        \end{bmatrix} $$
Menggunakan Ekspansi Laplace baris pertama untuk $A$:
$$ \det(A) = 1 \begin{vmatrix} 4 & 5 \\ 0 & 6 \end{vmatrix} - 2 \begin{vmatrix} 0 & 5 \\ 1 & 6 \end{vmatrix} + 3 \begin{vmatrix} 0 & 4 \\ 1 & 0 \end{vmatrix} $$
$$ = 1(24 - 0) - 2(0 - 5) + 3(0 - 4) = 24 + 10 - 12 = 22 $$
\end{frame}

\begin{frame}[t,fragile]{Contoh Determinan Transpose (lanjutan)}
Menggunakan Ekspansi Laplace baris pertama untuk $A^T$:
$$ A^T = \begin{bmatrix}
       	1 & 0 & 1 \\
	    2 & 4 & 0 \\
	    3 & 5 & 6
        \end{bmatrix} $$
$$ \det(A^T) = 1 \begin{vmatrix} 4 & 0 \\ 5 & 6 \end{vmatrix} - 0 \begin{vmatrix} 2 & 0 \\ 3 & 6 \end{vmatrix} + 1 \begin{vmatrix} 2 & 4 \\ 3 & 5 \end{vmatrix} $$
$$ = 1(24 - 0) - 0 + 1(10 - 12) = 24 - 2 = 22 $$
\vspace{0.3cm}
$$ \therefore \det(A) = \det(A^T) = 22 ~~ \checkmark $$
\end{frame}

\section{Sifat-sifat Determinan}
\begin{frame}[t,fragile]{Sifat-sifat Determinan (1)}
Berikut adalah sifat-sifat penting determinan:
\begin{enumerate}
\item \textbf{Determinan Transpose:} $\det(A^T) = \det(A)$
\item \textbf{Determinan Matriks Identitas:} $\det(I) = 1$
\item \textbf{Perkalian Skalar:} Jika matriks $A$ dikalikan dengan skalar $k$, maka:
$$ \det(kA) = k^n \det(A) $$
dimana $n$ adalah ukuran matriks (untuk matriks $n \times n$)
\item \textbf{Determinan Perkalian Matriks:} 
$$ \det(AB) = \det(A) \cdot \det(B) $$
\end{enumerate}
\end{frame}

\begin{frame}[t,fragile]{Sifat-sifat Determinan (2)}
\begin{enumerate}
\setcounter{enumi}{4}
\item \textbf{Baris/Kolom Nol:} Jika suatu baris atau kolom semuanya nol, maka $\det(A) = 0$
\item \textbf{Baris/Kolom Identik:} Jika dua baris atau dua kolom identik, maka $\det(A) = 0$
\item \textbf{Baris/Kolom Proporsional:} Jika suatu baris/kolom adalah kelipatan dari baris/kolom lain, maka $\det(A) = 0$
\item \textbf{Pertukaran Baris/Kolom:} Jika dua baris atau kolom ditukar, maka determinan berubah tanda:
$$ \det(A') = -\det(A) $$
\end{enumerate}
\end{frame}

\begin{frame}[t,fragile]{Sifat-sifat Determinan (3)}
\begin{enumerate}
\setcounter{enumi}{8}
\item \textbf{Perkalian Baris/Kolom dengan Skalar:} Jika suatu baris/kolom dikalikan dengan skalar $k$, determinan juga dikalikan dengan $k$
\item \textbf{Penjumlahan Kelipatan Baris/Kolom:} Jika suatu baris/kolom ditambah dengan kelipatan baris/kolom lain, determinan tidak berubah:
$$ R_i \rightarrow R_i + kR_j \Rightarrow \det(A') = \det(A) $$
\item \textbf{Determinan Matriks Segitiga:} Determinan matriks segitiga (atas atau bawah) adalah perkalian elemen diagonal utamanya
\item \textbf{Determinan Invers:} $\det(A^{-1}) = \frac{1}{\det(A)}$ (jika $A$ invertibel)
\end{enumerate}
\end{frame}

\begin{frame}[t,fragile]{Contoh Sifat Determinan}
\textbf{Contoh 1:} Determinan perkalian matriks
$$ A = \begin{bmatrix} 2 & 1 \\ 3 & 4 \end{bmatrix}, \quad B = \begin{bmatrix} 1 & 2 \\ 0 & 1 \end{bmatrix} $$
$$ \det(A) = 8 - 3 = 5, \quad \det(B) = 1 - 0 = 1 $$
$$ AB = \begin{bmatrix} 2 & 5 \\ 3 & 10 \end{bmatrix}, \quad \det(AB) = 20 - 15 = 5 $$
$$ \det(AB) = \det(A) \cdot \det(B) = 5 \cdot 1 = 5 ~~ \checkmark $$
\end{frame}

\begin{frame}[t,fragile]{Contoh Sifat Determinan (lanjutan)}
\textbf{Contoh 2:} Determinan matriks segitiga atas
$$ A = \begin{bmatrix}
       	2 & 3 & 1 \\
	    0 & 4 & 5 \\
	    0 & 0 & 6
        \end{bmatrix} $$
Determinan = perkalian elemen diagonal:
$$ \det(A) = 2 \times 4 \times 6 = 48 $$
\vspace{0.3cm}
\textbf{Contoh 3:} Perkalian skalar
$$ A = \begin{bmatrix} 1 & 2 \\ 3 & 4 \end{bmatrix}, \quad \det(A) = -2 $$
$$ 2A = \begin{bmatrix} 2 & 4 \\ 6 & 8 \end{bmatrix}, \quad \det(2A) = 16 - 24 = -8 = 2^2 \cdot (-2) ~~ \checkmark $$
\end{frame}

\section{Persamaan Linear Simultan}
\begin{frame}[t,fragile]{Sistem Persamaan Linear}
Sistem persamaan linear dengan $n$ variabel dan $n$ persamaan dapat ditulis dalam bentuk matriks:
$$ AX = B $$
dimana:
\begin{itemize}
\item $A$ adalah matriks koefisien ($n \times n$)
\item $X$ adalah vektor variabel ($n \times 1$)
\item $B$ adalah vektor konstanta ($n \times 1$)
\end{itemize}
\vspace{0.3cm}
\textbf{Contoh:} Sistem 2 persamaan 2 variabel:
\begin{align*}
2x + 3y &= 8 \\
x - y &= -1
\end{align*}
dapat ditulis: $\begin{bmatrix} 2 & 3 \\ 1 & -1 \end{bmatrix} \begin{bmatrix} x \\ y \end{bmatrix} = \begin{bmatrix} 8 \\ -1 \end{bmatrix}$
\end{frame}

\begin{frame}[t,fragile]{Metode Invers Matriks}
Jika $\det(A) \neq 0$, maka sistem $AX = B$ memiliki solusi tunggal:
$$ X = A^{-1}B $$
\vspace{0.3cm}
Untuk matriks $2 \times 2$: $A = \begin{bmatrix} a & b \\ c & d \end{bmatrix}$
$$ A^{-1} = \frac{1}{\det(A)} \begin{bmatrix} d & -b \\ -c & a \end{bmatrix} = \frac{1}{ad - bc} \begin{bmatrix} d & -b \\ -c & a \end{bmatrix} $$
\vspace{0.3cm}
\textbf{Catatan:}
\begin{itemize}
\item Jika $\det(A) = 0$, sistem tidak memiliki solusi tunggal
\item Sistem bisa tidak memiliki solusi atau memiliki tak hingga solusi
\end{itemize}
\end{frame}

\begin{frame}[t,fragile]{Contoh Metode Invers}
Selesaikan sistem: $\begin{cases} 2x + 3y = 8 \\ x - y = -1 \end{cases}$
\vspace{0.2cm}
$$ A = \begin{bmatrix} 2 & 3 \\ 1 & -1 \end{bmatrix}, \quad B = \begin{bmatrix} 8 \\ -1 \end{bmatrix} $$
$$ \det(A) = 2(-1) - 3(1) = -2 - 3 = -5 $$
$$ A^{-1} = \frac{1}{-5} \begin{bmatrix} -1 & -3 \\ -1 & 2 \end{bmatrix} = \begin{bmatrix} 1/5 & 3/5 \\ 1/5 & -2/5 \end{bmatrix} $$
$$ X = A^{-1}B = \begin{bmatrix} 1/5 & 3/5 \\ 1/5 & -2/5 \end{bmatrix} \begin{bmatrix} 8 \\ -1 \end{bmatrix} = \begin{bmatrix} 8/5 - 3/5 \\ 8/5 + 2/5 \end{bmatrix} = \begin{bmatrix} 1 \\ 2 \end{bmatrix} $$
Jadi, $x = 1$ dan $y = 2$
\end{frame}

\begin{frame}[t,fragile]{Aturan Cramer}
\textbf{Aturan Cramer} menyatakan bahwa untuk sistem $AX = B$, solusinya adalah:
$$ x_i = \frac{\det(A_i)}{\det(A)} $$
dimana $A_i$ adalah matriks yang diperoleh dengan mengganti kolom ke-$i$ dari $A$ dengan vektor $B$.
\vspace{0.3cm}
Untuk sistem 2 variabel:
$$ x = \frac{\det(A_1)}{\det(A)}, \quad y = \frac{\det(A_2)}{\det(A)} $$
dimana:
$$ A_1 = \begin{bmatrix} b_1 & a_{12} \\ b_2 & a_{22} \end{bmatrix}, \quad A_2 = \begin{bmatrix} a_{11} & b_1 \\ a_{21} & b_2 \end{bmatrix} $$
\end{frame}

\begin{frame}[t,fragile]{Contoh Aturan Cramer}
Selesaikan sistem: $\begin{cases} 2x + 3y = 8 \\ x - y = -1 \end{cases}$ dengan Aturan Cramer
\vspace{0.2cm}
$$ A = \begin{bmatrix} 2 & 3 \\ 1 & -1 \end{bmatrix}, \quad \det(A) = -5 $$
$$ A_1 = \begin{bmatrix} 8 & 3 \\ -1 & -1 \end{bmatrix}, \quad \det(A_1) = 8(-1) - 3(-1) = -8 + 3 = -5 $$
$$ A_2 = \begin{bmatrix} 2 & 8 \\ 1 & -1 \end{bmatrix}, \quad \det(A_2) = 2(-1) - 8(1) = -2 - 8 = -10 $$
$$ x = \frac{\det(A_1)}{\det(A)} = \frac{-5}{-5} = 1 $$
$$ y = \frac{\det(A_2)}{\det(A)} = \frac{-10}{-5} = 2 $$
Jadi, $x = 1$ dan $y = 2$ (sama dengan metode invers)
\end{frame}

\begin{frame}[t,fragile]{Ringkasan Metode Penyelesaian Sistem Linear:}
\begin{enumerate}
\item \textbf{Metode Invers Matriks:} $X = A^{-1}B$
    \begin{itemize}
    \item Efisien untuk banyak sistem dengan matriks $A$ yang sama
    \item Memerlukan perhitungan invers matriks
    \end{itemize}
\item \textbf{Aturan Cramer:} $x_i = \frac{\det(A_i)}{\det(A)}$
    \begin{itemize}
    \item Baik untuk sistem kecil (2-3 variabel)
    \item Memerlukan perhitungan banyak determinan
    \end{itemize}
\item \textbf{Eliminasi Gauss:} (akan dipelajari lebih lanjut)
    \begin{itemize}
    \item Paling efisien untuk sistem besar
    \item Tidak memerlukan perhitungan determinan
    \end{itemize}
\end{enumerate}
\end{frame}

\begin{frame}[t,fragile]{Ringkasan Determinan}
\begin{itemize}
\item Determinan adalah bilangan yang terkait dengan matriks bujur sangkar
\item Minor dan kofaktor digunakan untuk menghitung determinan
\item Ekspansi Laplace: metode umum untuk semua orde
\item Aturan Sarrus: metode cepat khusus untuk $3 \times 3$
\item $\det(A) = \det(A^T)$
\item Sifat-sifat determinan berguna untuk perhitungan efisien
\item Determinan digunakan dalam:
    \begin{itemize}
    \item Menentukan invertibilitas matriks
    \item Aturan Cramer untuk sistem linear
    \item Menghitung invers matriks
    \end{itemize}
\end{itemize}
\end{frame}

\end{document}
