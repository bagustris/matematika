\documentclass{beamer}
\usepackage[utf8]{inputenc}
\usepackage[bahasa]{babel}
\usepackage{amsmath}
\usepackage{breqn}

\usecolortheme{beaver}
\title{Bilangan Kompleks}
\author{@btatmaja\\Dept. of Engineering Physics \\ Institut Teknologi Sepuluh Nopember}
\date{\today}
\begin{document}
	\frame{\titlepage}
	\frame{
		\frametitle{Table of Contents}
		\tableofcontents
	}

\section{0. Bilangan}
\begin{frame}[t,fragile]{Bilangan}
\begin{itemize}
\item Bilangan Real
\begin{itemize}
\item Bil. Asli, $1, 2, 3, ...$ \vskip10pt
\item Bil. Bulat, $...,-2,-1,0,1,2,...$ \vskip5pt
\item Bil. Rasional, $\cfrac{1}{2}, \cfrac{1}{3},\cfrac{2}{5}, ...$ \vskip10pt
\item Bil. Irasional, $\sqrt{2}, \sqrt{3}, \sqrt{5}$ \vskip10pt
\end{itemize}
\item Bilangan Kompleks \\
\hskip10pt $z=a+bi$
\end{itemize}
\end{frame}

\section{1. Bentuk kutub dari bilangan kompleks}
\begin{frame}[t, fragile]{Bilangan Kompleks}
\begin{center}
\fbox{\huge{$Z = a + bi$}}
\end{center}
\vskip10pt
\begin{itemize}
\item $i=\sqrt{-1}$ (satuan imaginer)
\item $i^2 = -1$
\item a bagian real dari z, ditulis  Re z =a
\item b bagian imaginer dari z, ditulis, Im z =b
\end{itemize}
\end{frame}

\begin{frame}[t, fragile]{Bilangan Kompleks}
Diberikan dua bilangan kompleks: $Z_1=a+bi$\\
                     \hskip160pt $Z_2=c+di$, maka:
\vskip10pt
\begin{enumerate}
\item $z_1+z_2 = (a+bi) + (c+di) = (a+c) + (b+d)i$
\item $z_1-z_2 = (a+bi) - (c+di) = (a-c) + (b-d)i$
\item $z_1 z_2 = (a+bi) (c+di) = (ac-bd) + (ad + bc)i$
\item $\cfrac{z_1}{z_2}= \cfrac{a+bi}{c+di}=\cfrac{ac+bd}{c^2+d^2} + \cfrac{(bc - ad)i}{c^2+d^2}$
\end{enumerate}
\end{frame}

\begin{frame}[t, fragile]{Bentuk kutub dari bilangan kompleks}
\begin{columns}
\column{.5\textwidth}
\includegraphics[width=2in]{pict/kompleks1.png}
\column{.5\textwidth}
Bidang XOY = bidang kompleks \\
$z = a + bi \rightarrow r=\sqrt{x^2+y^2}$ \\
r disebut modulus dari nilai z atau \\
Nilai mutlak dari z, ditulis $|z|$ \\
$\sin \theta=\cfrac{b}{r} \longrightarrow$  \hskip10pt $\theta$ disebut \\
$\cos \theta=\cfrac{a}{r}$  \hskip30pt argumen dari z\\
\end{columns}
\begin{center}
    $z=a+bi \rightarrow$ \fbox{$z=r(\cos \theta + i \sin \theta)$}
\end{center}
\begin{exampleblock}{Soal:}
Nyatakan $z =  1 + \sqrt{3} i$  ke dalam bentuk kutub
\end{exampleblock}
\end{frame}

\section{2. Conjugate}
\begin{frame}[t, fragile]{Conjugate}
Conjugate dari $z=a+bi$ ialah \underline{$\bar{z}=a-bi$}\\
\vskip5pt
\begin{columns}
\column{.5\textwidth}
\begin{itemize}
\item $z=a+bi$
\item $\bar{z}=\overline{a+bi}$
\item $z_1=a+bi$
\item $z_2=c+di$
\end{itemize}
\column{.5\textwidth}
\begin{enumerate}
\item $\bar{\bar{z}}=z$
\item $z.\bar{z}=|z|^2=|\bar{z}|^2$
\item $\overline{z_1\pm z_2}=\bar{z_1}\pm \bar{z_2}$
\item $\overline{z_1 z_2} = \bar{z_1} . \bar {z_2}$
\item $\overline{\frac{z_1}{z_2}} = \frac{\bar{z_1}}{\bar{z_2}}$
\end{enumerate}
\end{columns}
Jika: \\ 
\hskip30pt \fbox{$ z_1=r_1(\cos \theta_1+i \sin \theta_1)$ dan $z_2=r_2(\cos \theta_2+i \sin \theta_2$ )}\\
maka:\\
\begin{enumerate}
\item $z_1 z_2 = r_1 r_2[\cos (\theta_1+\theta_2) + i \sin (\theta_1+\theta_2)]$
\item $\cfrac{z_1}{z_2}=\cfrac{r_1}{r_2} \bigl[{\cos (\theta_1-\theta_2) + i \sin (\theta_1-\theta_2)}\bigr]$
\end{enumerate}
\end{frame}

\section{3. Teorema De Moivre}
\begin{frame}[t, fragile]{Teorema De Moivre}
Abraham De Moivre (1667-1754) menyatakan untuk setiap bilangan rasional n berlaku:\\
\vskip10pt
\hskip10pt \fbox{$[r (\cos \theta + i \sin \theta)]^n = r^n (\cos (n\theta) + i \sin (n\theta))$}
\vskip10pt
Jika r=1 maka,
\vskip10pt
\hskip10pt \fbox{$(\cos \theta + i \sin \theta)^n = \cos (n\theta) + i \sin (n\theta)$}
\vskip10pt
\begin{exampleblock}{Contoh}
Dapatkan nilai dari $(\sqrt{3}+i)^6$
\end{exampleblock}
\end{frame}

\section{4. Penarikan Akar}
\begin{frame}[t, fragile]{Penarikan akar}
\begin{itemize}
\item a + bi = r $(\cos \theta + i \sin \theta)$ \\
Karena 
\begin{align}
     \sin \theta& = \sin (\theta + k.360^o) \rightarrow  k= bil. bulat \\
     \cos \theta& = \cos (\theta + k.360^o) 
\end{align}
\item maka: a + bi = $r [\cos(\theta + k.360^o) + i \sin(\theta+ k.360^o)]$
\item Jika $z^n = a + bi \rightarrow z_{1,2,3,..,n}= \sqrt[n]{a+bi} = ....?$
\item Penyelesaiannya:
\end{itemize}
\hskip22pt \fbox{$z_{1,2,3,...,n}=r^{\frac{1}{n}}\biggl[\cos{\dfrac{\theta+k.360^o}{n}} + i \sin{\dfrac{\theta+k.360^o}{n}} \biggr]$}
\end{frame}

\end{document}
