\documentclass{beamer}
\usepackage[utf8]{inputenc}
\usepackage{graphicx}
\title{Aljabar Vektor}
\author{@btatmaja\\Institut Teknologi Sepuluh Nopember}
\date{\today}
\begin{document}
	\frame{\titlepage}
	\frame{
		\frametitle{Table of Contents}
		\tableofcontents
	}

\section{Besaran vektor}
\begin{frame}[t,fragile]{Besaran Vektor}
\begin{block}{Besaran}
\end{block}
\begin{itemize}
\item Vektor: Kuantiti yang memiliki besar dan arah \\
Contoh: kecepatan, percepatan, gaya.
\item Skalar: Kuantiti yang hanya memiliki besar saja.\\
Contoh: waktu, temperatur, massa, panjang.
\end{itemize}
\includegraphics[width=1.8in]{pict/vektor.png}
$\overrightarrow{AB} = \vec{a}; ~|\vec{a}| = a =$ panjang vektor $\vec{a}$
\end{frame}

\section{Dua vektor sama}
\begin{frame}[t, fragile]{Dua vektor sama}
$\vec{a} = \vec {b} \Longrightarrow $ searah, sama panjang \\
\begin{center}
\includegraphics[width=4in]{pict/duavektor.png} \\
\end{center}
$\overrightarrow{-a}$, sama panjang tetapi berlawanan arah dengan $\vec{a}$
\end{frame}

\section{Jumlah dua vektor: $\vec{a} + \vec{b}$}
\begin{frame}[t, fragile]{Jumlah dua vektor: $\vec{a} + \vec{b}$}
$\longrightarrow \vec{b}$ dimulai dari ujung vektor $\vec{a}$, vektor $\vec{b} ~\# ~\vec{a}$, lalu hubungkan pangkal $\vec{a}$ dengan ujung $\vec{b}$ tsb.
\begin{center}
\includegraphics[width=4in]{pict/jumlahvektor} \\
\end{center}
\begin{center}
\fbox{$\vec{a}+\vec{b}=\vec{b}+\vec{a}$}
\end{center}
\end{frame}

\section{Selisih dua vektor: $\vec{a} - \vec{b}$}
\begin{frame}[t, fragile]{Selisih dua vektor: $\vec{a} - \vec{b}$}
$$ \vec{a} - \vec {b} = \vec{a} + (-\vec{b}) = \vec{a}~~ ditambah~~ -\vec{b}$$
\begin{center}
\includegraphics[width=4in]{pict/selisihvektor} \\
\end{center}
\begin{center}
Jika $\vec{a}=\vec{b} \Longrightarrow \vec{a}-\vec{b}=\vec{0}$ \\
Vektor nol, panjang nol, arah tak didefinisikan
\end{center}
\end{frame}

\section{Vektor satuan}
\begin{frame}[t, fragile]{Vektor Satuan}
Vektor satuan $\vec{i}$: Vektor dari titik (0,0) sampai titik (1,0).\\
Vektor satuan $\vec{j}$: Vektor dari titik (0,0) sampai titik (0,1).\\
\begin{center}
\includegraphics[width=3in]{pict/unitVektor.png}
\end{center}
$\vec{a}=a_1 \vec{i} + a_2 \vec{j}$, Contoh: $\vec{a}=3i+2j$
\begin{center}
\includegraphics[width=1.4in]{pict/unitVektor2.png}
\end{center}
\end{frame}


\begin{frame}[t, fragile]{Unit vektor siku-siku}
$|\vec{i}| = |\vec{j}|=|\vec{k}|=1$
\begin{center}
\includegraphics[width=4in]{pict/vektorsiku.png}
\end{center}
\hspace{180pt} $\vec{a}=a_1\vec{i}+a_2\vec{j}+a_3\vec{k}$ \\
Vektor posisi $\vec{r}$ dari O ke $P(x,y,z)$ adalah:\\
$\vec{r} = x\vec{i} + y\vec{j} + z\vec{k}$ \\
dengan panjang $r=\sqrt{x^2+y^2+z^2}; ~~a=|\vec{a}|=\sqrt{a_1^{2}+a_2^{2}+a_3^{2}}$
\end{frame}

\section{Perkalian titik (Dot product)}
\begin{frame}[t, fragile]{Perkalian titik (Dot product)}
%\begin{center}
\begin{columns}
\column{.6\textwidth}
Definisi: \fbox{$\vec{a}.\vec{b} = ab \cos \theta$}; ~~$(0\leq\theta\leq\pi)$ \\
\vskip3pt $\vec{a}.\vec{b} = \vec{b}.\vec{a} \Longrightarrow \vec{i}.\vec{i} =\vec{j}.\vec{j} =\vec{k}.\vec{k} =1$\\
\vskip3pt \hskip67pt                                         $\vec{i}.\vec{j} =\vec{j}.\vec{k} =\vec{k}.\vec{i} =0$
\column{.3\textwidth}
\includegraphics[width=1.5in]{pict/dotproduct} \\
\end{columns}
\vskip5pt
$\vec{a}=a_1\vec{i}+a_2\vec{j}+a_3\vec{k} \Longrightarrow$ \fbox{$\vec{a}.\vec{b}=a_1b_1+a_2b_2+a_3b_3$} \\
$\vec{b}=b_1\vec{i}+b_2\vec{j}+b_3\vec{k}$ \hskip70pt skalar\\
$$\cos \theta = \dfrac {\vec{a}\vec{b}} {ab} = \dfrac{a_1b_1+a_2b_2+a_3b_3}{\sqrt{a_1^2+a_2^2+a_3^2}.\sqrt{b_1^2+b_2^2+b_3^2}}$$
\end{frame}

\section{Perkalian silang (Cross product)}
\begin{frame}[t, fragile] {Perkalian silang (Cross product)}
\begin{columns}
\column{.3\textwidth}
\includegraphics[width=1.5in]{pict/crossproduct}
\vskip10pt
\fbox{$\vec{a}\times\vec{b}= -(\vec{b}\times \vec{a})$}
\column{.65\textwidth}
\underline{Definisi:} \\
\vskip3pt
$\vec{a}\times\vec{b} = (ab \sin{\theta} )~\vec{e}; ~~~0\leq \theta \leq \pi$\\
\vskip3pt
$\theta = \angle(\vec{a}, \vec{b})$ diukur dari $\vec{a}$ ke $\vec {b}$ \\
\vskip3pt
$\vec{e}$ = vektor satuan $\perp$ bidangnya $\vec{a}$ dan $\vec {b}$. \\
\end{columns}
\vskip20pt
$$\vec{i} \times \vec{i} = \vec{j} \times \vec{j} = \vec{k} \times \vec{k}=0$$
$$\vec{i}\times \vec{j} =\vec{k};~~~ \vec{j}\times \vec{k} =\vec{i};~~~ \vec{k}\times\vec{i} = \vec{j};$$
$$\vec{j}\times \vec{i} = -\vec{k};~~~ \vec{k}\times \vec{j} =-\vec{i};~~~ \vec{i}\times\vec{k} = -\vec{j};$$
\end{frame}

\begin{frame}[t,fragile]{Beberapa rumus[1]}
\begin{itemize}
\item{Luas jajaran gejang yang dibentuk $\vec{a}$ dan $\vec{b}$}
\includegraphics[width=1.5in]{pict/jajar_genjang}
\item{ Penulisan disingkat}
\end{itemize}
\end{frame}

\begin{frame}[t,fragile]{Beberapa rumus[2]}
\begin{itemize}
\item{Tiga vektor $\vec{a} ~~ \vec{b} ~~ \vec{c}$ membentuk paralelepipedum (balok miring)}
\item{Rumus} \\
\begin{center}
\fbox{$\vec{a} \times (\vec{b} \times \vec{c}) = (\vec{a} \cdot \vec{c}) - (\vec{a} \cdot \vec{b}) ~\vec{c}$}
\end{center}
\end{itemize}
\end{frame}

\begin{frame}[t,fragile]{Contoh Soal}
\begin{itemize}
\item{Dapatkan sudut antara 2 vektor: $ \vec{a}= 2 \vec{i} + 2 \vec{j} - \vec{k} ~~dan~~
					\vec{b}= 6 \vec{i} - 3 \vec{j} - 2 \vec{k}$}
\item{Jawab:} \\
\vspace{5px}
$\vec{a} = \sqrt{(2)^{2} + (2)^{2}+(-1)^2} = 1;$ \\
$\vec{b} = \sqrt{(6)^{2} + (-3)^{2}+(2)^2} = 1;$ \\
$\vec{a} \cdot \vec{b} = (2) (6) + (2) (-3) + (-1) (2) = 4$ \\
$\cos~\theta = \dfrac{\vec{a}\vec{b}}{ab} =\dfrac{4}{(3)(7)} =0.1905 ~ \Rightarrow ~ \theta = 79^{o}$
\end{itemize}
\end{frame}

\begin{frame}[t,fragile]{Soal}
\begin{enumerate}
\item Diketahui vektor $\vec{a} = 3i + 4j$ dan vektor $\vec{b} = 2i + j$. Hitunglah harga-harga :
\begin{itemize} 
\item $\vec{a} + \vec{b}$ 
\item $\vec b + \vec a$
\item $\vec a ~– \vec b$
\item $\vec b ~– \vec a$
\item $|\vec a|$ dan $|\vec b|$ 
\item sudut $\vec a$ 
\item sudut $\vec b$
\item $\vec a \cdot \vec b$ 
\item $\vec a \times \vec b$
\end{itemize}
\item Diketahui vektor-vektor $\vec a , \vec{b}$ dan $\vec{c}$ seperti di bawah ini. Lukislah secara grafis operasi vektor :$ \vec a - \vec{b} +2. \vec c$ dan $3 \vec c - 0,5(2 \vec a - \vec b )$.\\
\centering \includegraphics[width=1.3in]{pict/soal2}
\end{enumerate}
\end{frame}

\begin{frame}[t,fragile]{Cartesian System}
\begin{itemize}
\item{a}
\item{b}
\end{itemize}
\end{frame}


\end{document}
