\documentclass{beamer}
\usepackage[utf8]{inputenc}
\usepackage[bahasa]{babel}
\usecolortheme{beaver}
\title{Bilangan Kompleks}
\author{@btatmaja\\Dept. of Engineering Physics \\ Institut Teknologi Sepuluh Nopember}
\date{\today}
\begin{document}
	\frame{\titlepage}
	\frame{
		\frametitle{Table of Contents}
		\tableofcontents
	}
	
\section{1. Bentuk kutub dari bilangan kompleks}
\begin{frame}[t, fragile]{Bilangan Kompleks}
\begin{center}
\fbox{\huge{$Z = a + bi$}}
\end{center}
\vskip10pt
\begin{itemize}
\item $i=\sqrt{-1}$ (satuan imaginer)
\item $i^2 = -1$
\item a bagian real dari z, ditulis  Re z =a
\item b bagian imaginer dari z, ditulis, Im z =b
\end{itemize}
\end{frame}

\begin{frame}[t, fragile]{Bilangan Kompleks}
Diberikan dua bilangan kompleks: $Z_1=a+bi$\\
                     \hskip160pt $Z_2=c+di$, maka:
\vskip10pt
\begin{enumerate}
\item $z_1+z_2 = (a+bi) + (c+di) = (a+c) + (b+d)i$
\item $z_1-z_2 = (a+bi) - (c+di) = (a-c) + (b-d)i$
\item $z_1 z_2 = (a+bi) (c+di) = (ac-bd) + (ad + bc)i$
\item $\cfrac{z_1}{z_2}= \cfrac{a+bi}{c+di}=\cfrac{ac+bd}{c^2+d^2} + \cfrac{bc - ad}{c^2+d^2}i$
\end{enumerate}
\end{frame}

\begin{frame}[t, fragile]{Bentuk kutub dari bilangan kompleks}
\begin{columns}
\column{.5\textwidth}
\includegraphics[width=2in]{pict/kompleks1.png}
\column{.5\textwidth}
Bidang XOY = bidang kompleks \\
$z = a + bi \rightarrow r=\sqrt{x^2+y^2}$ \\
r disebut modulus dari nilai z atau \\
Nilai mutlak dari z, ditulis $|z|$ \\
$\sin \theta=\cfrac{b}{r} \longrightarrow$  \hskip10pt $\theta$ disebut \\
$\cos \theta=\cfrac{a}{r}$  \hskip30pt argumen dari z\\
\end{columns}
\begin{center}
    $z=a+bi \rightarrow$ \fbox{$z=r(\cos \theta + i \sin \theta)$}
\end{center}
\begin{exampleblock}{Soal:}
Nyatakan $z =  1 + \sqrt{3} i$  ke dalam bentuk kutub
\end{exampleblock}
\end{frame}

\begin{frame}[t, fragile]{Conjugate}
Conjugate dari $z=a+bi$ ialah \underline{$\bar{z}=a-bi$}\\
\vskip5pt
\begin{columns}
\column{.5\textwidth}
\begin{itemize}
\item $z=a+bi$
\item $\bar{z}=\bar{a+bi}$
\item $z_1=a+bi$
\item $z_2=c+di$
\end{itemize}
\column{.5\textwidth}
\begin{enumerate}
\item $\bar{\bar{z}}=z$
\item $z.\bar{z}=|z|^2=|\bar{z}|^2$
\item $\bar{z_1\pm z_2}=\bar{z_1}\pm{z_2}$
\item $\bar{z_1 z_2} = \bar{z1} . \bar {z_2}$
\item $\bar{\cfrac{z_1}{z_2}} = \cfrac{\bar{z_1}}{\bar{z_2}}$
\end{enumerate}
\end{columns}
Jika: \\ 
\hskip30pt \fbox{$ z_1=r_1(\cos \theta_1+i \sin \theta_1)$ dan $z_2=r_2(\cos \theta_2+i \sin \theta_2$ )}\\
maka:\\
\begin{enumerate}
\item $z_1 z_2 = r_1 r_2[\cos (\theta_1+\theta_2) + i \sin (\theta_1+\theta_2)]$
\item $\cfrac{z_1}{z_2}=\cfrac{r_1}{r_2} \bigl[{\cos (\theta_1+\theta_2) + i \sin (\theta_1+\theta_2)}\bigr]$
\end{enumerate}
\end{frame}

\section{2. Teorema De Moivre dan Penarikan Akar}
\begin{frame}[t, fragile]{Teorema De Moivre}
\end{frame}

\end{document}
