\documentclass{beamer}
\usepackage[utf8]{inputenc}
\usepackage[bahasai]{babel}
\usepackage{amsmath}
\title{Matriks}
\author{Bagus Tris Atmaja\\Institut Teknologi Sepuluh Nopember}
\date{\today}
\begin{document}
	\frame{\titlepage}
	\frame{
		\frametitle{Table of Contents}
		\tableofcontents
	}
	
\section{Definisi-definisi matriks}
\begin{frame}[t,fragile]{Definisi}
Matriks adalah suatu susunan deretan empat persegi panjang dari bilangan-bilangan yang tersusun dalam $m$ baris dan $n$ kolom dalam tanda kurung.\\
Contoh:\\

$$A = \begin{bmatrix}
  a_{11} & a_{12} & a_{13} & a_{14}\\
  a_{21} & a_{22} & a_{23} & a_{24}\\
  a_{31} & a_{32} & a_{33} & a_{34}
 \end{bmatrix} $$

adalah suatu matriks berukuran 3x4. Ditulis $A_{3x4}$ (3 baris 4 kolom).\\
Matriks bujur sangkar adalah suatu matriks yang jumlah barisnya sama dengan jumlah kolomnya (disebut matriks dengan orde $n$).\\
Matriks Simetrik adalah matrik bujur sangkar jika ditransposekan hasilnya sama.\\
$$A = A^T$$
\end{frame}

\begin{frame}[t,fragile]{Matriks}
$$A = \begin{bmatrix}
    2 & 1 & 3 \\
    1 & 5 & 0 \\
    3 & 0 & 1
   \end{bmatrix} $$

\begin{itemize}
\item \underline{Trace}: Jumlah elemen diagonal pokok suatu matriks bujur sangkar.\\
$$Trace~~A = 2 + 5 + 1 = 8$$
\item \underline{Matriks singular}: Jika determinan matriks bujur sangkar adalah nol.
\item \underline{Matriks baris}: Jika $m=1$\\
$$ B = \bigl[ b_1 ~~~ b_2 ~~~ b_3 \bigr]$$ 
\end{itemize}
\end{frame}

\begin{frame}[t,fragile]{Matriks}
\begin{itemize}
\item \underline{Matriks kolom}: Jika $n=1$ \\
$$ C = \begin{bmatrix}
       	C_1 \\
	    C_2 \\
	    C_3 
        \end{bmatrix} $$
\item \underline{Matris nol}: matrik yang semua elemennya adalah nol.\\
$$ B = \begin{bmatrix}
       	0 & 0 & 0\\
	    0 & 0 & 0 
        \end{bmatrix} $$
\item \underline{Matriks Satuan}:\\
$$ C = \begin{bmatrix}
       	1 & 0 & 0 \\
	    0 & 1 & 0 \\
	    0 & 0 & 1
        \end{bmatrix} $$
\end{itemize}
\end{frame}

\begin{frame}[t,fragile]{Matriks}
\begin{itemize}
\item \underline{Matriks Diagonal}: Matriks bujur sangkar yang mempunyai nilai pada elemen diagonal sedang elemen lainnya nol.
$$ a_{ii} = \begin{bmatrix}
       	a_{11} & 0 & 0 \\
	    0 & a_{12} & 0 \\
	    0 & 0 & a_{13} 
        \end{bmatrix} $$
\item \underline{Transpose}: transpose dari matriks A adalah matriks dimana baris dan kolom dipertukarkan.
\begin{columns}
\column{.5\textwidth}
$$ A = \begin{bmatrix}
       	a_{11} & a_{12} & a_{13}\\
	    a_{21} & a_{22} & a_{23}\\
        \end{bmatrix} $$
\column{.5\textwidth}
$$ \Longrightarrow A^T = \begin{bmatrix}
       	a_{11} & a_{21}\\
	    a_{12} & a_{22}\\
        a_{13} & a_{23} 
        \end{bmatrix} $$
\end{columns}
\item \underline{Minor}: Minor $M_{ij}$ dari suatu matriks A dibentuk dengan menghilangkan baris ke $i$ dan kolom ke $j$ dari matriks asal.
\begin{columns}
\column{.5\textwidth}
$$ A = \begin{bmatrix}
       	a_{11} & a_{12} & a_{13}\\
	    a_{21} & a_{22} & a_{23}\\
        a_{31} & a_{32} & a_{33} 
        \end{bmatrix} $$
\column{.5\textwidth}
$$ \Longrightarrow M_{12} = \begin{bmatrix}
       	a_{21} & a_{21}\\
        a_{31} & a_{33} 
        \end{bmatrix} $$
\end{columns}
\end{itemize}
\end{frame}

\begin{frame}[t,fragile]{Matriks}
\begin{itemize}
\item \underline{Kofaktor}: Kofaktor $C_{ij}$ adalah sama dengan minor yang tertanda $(-1)^{i+j} M_{ij}$.\\
    $$ C_{12} = (-1)^{1+2} ~ M_{12} = - M_{12} $$
\item \underline{Matriks Inversi}: Inversi $A^{(-1)}$ dari matriks $A$ memenuhi hubungan,\\ 
    $$ A^{-1} A = A A^{-1} = I $$
\item \underline{Matriks orthogonal}: Matriks yang mememenuhi hubungan,\\
    $$ A^T A = A A^T = I$$ 
\end{itemize}
\end{frame}

\begin{frame}[t,fragile]{Adjoint Matriks}
Suatu matriks adjoint dari matriks bujur sangkar $A$ adalah transpose matriks kofaktor $A$. \\
Misalkan matriks kofaktor $A$ adalah:
$$ C_{ij} = \begin{bmatrix}
            C_{11} & C_{12} & C_{13} \\
            C_{21} & C_{22} & C_{23} \\
            C_{31} & C_{32} & C_{33}
            \end{bmatrix} $$
$$ Adj~A = \bigl[ C_{ij}^T \bigr] = \bigl[ C_{ji} \bigr] =  \begin{bmatrix}
                                                            C_{11} & C_{21} & C_{31} \\
                                                            C_{12} & C_{22} & C_{32} \\
                                                            C_{13} & C_{23} & C_{33}
                                                            \end{bmatrix} $$
\end{frame}

\section{Aturan Operasi Matriks}	
\begin{frame}[t,fragile]{Operasi Matriks}
\begin{itemize}

\end{itemize}
\end{frame}


\section{Ekspansi Laplace}

\section{Aturan Sarrus}

\section{Determinan Transpose}

\section{Sifat-sifat Determinan}

\section{Persamaan Linear Simultan}

\end{document}
