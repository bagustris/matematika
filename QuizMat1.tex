\documentclass[twocolumn]{article}
\usepackage[utf8]{inputenc}
\usepackage[bahasai]{babel}
\usepackage{graphicx}
\usepackage{float}

\title{Quiz Matematika 1}
\author{}
\date{3 Oktober 2016}
\twocolumn
\begin{document}
\maketitle
\begin{enumerate}
\item Buatlah grafik berikut:
\begin{enumerate}
\item $y=|2x-4|+1$
\item $y=|3-x|-1$
\item $y=|\sin 6x|$
\end{enumerate}
\item Tentukan persamaan dari:
\begin{figure}[H]
\includegraphics[width=2in]{pict/gambar2a.png}
\end{figure}
\begin{figure}[H]
\includegraphics[width=2in]{pict/gambar2b.png}
\end{figure}
\item Berapakah jarak antara kedua garis:\\
$ax+by+c=0$ dan $ax+by+D=0$
\item Hitunglah:
\begin{enumerate}
\item $I=\int\limits_{-6}^4 |2-x|dx$
\item $I=\int\limits_{-4}^{10}|-x^2-2x+3|dx$
\item $I=\int\limits_{0}^\pi|\cos{3x}|dx$
\end{enumerate}
\item Tentukan gambar parabola:
\begin{enumerate}
\item $y=x^2-2x+8$
\item $t=4-3x-x^2$
\item $y=x^2+1$
\end{enumerate}
\item A(1,0); B(5,3); C(7,6) dan D(3,4). Dapatkan luas ABCD.
\item Tentukan persamaan parabola dengan puncak (4, 16) dan melewati titik (2,0).
\item Tentukan persamaan parabola berikut.
\begin{figure}[H]
\includegraphics[width=2in]{pict/parabolaQuiz.png}
\end{figure}
\end{enumerate}
\end{document}
