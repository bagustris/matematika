\documentclass{article}
\usepackage[utf8]{inputenc}
\usepackage[bahasai]{babel}
\usepackage{graphicx}
\usepackage{float}
\usepackage{amsmath}

\title{Fungsi}
\author{Y. Susatio \& B.T. Atmaja}
\date{\today}

\begin{document}
\maketitle
\section{Definisi Fungsi}
Diberikan $y=f(x) \Longrightarrow$  dibaca: $y$ adalah fungsi dari $x$. $X$ disebut peubah bebas, $y$ disebut peubah tak bebas. $y$ dikatakan sebagai fungsi $x$, jika 1 harga $x$ menentukan 1 harga $y$.\\
$y=x^2\rightarrow$ karena 1 buah harga menentukan satu buah $y$\\
$y^2=x\rightarrow$ $y$ bukan fungsi $x$, karena 1 buah harga $x$ dapan menghasilkan 2 buah harga $y$.
\begin{center}
\includegraphics[width=4in]{pict/fungsi1}
\end{center}

\begin{center}
%\includegraphics{width=]{pict/}
\end{center}
\section{Daerah Definisi dan Daerah Fungsi}
Contoh: Dapatkan daerah definisi dan daerah fungsi dari:\\
a) $y=4x^2$       \hspace{32pt} b) $y=2x^2+1$ \\
c) $y=x^2-4$      \hspace{20pt} d) $y=\sqrt{9-x^2} $\\
e) $y=^2 logx $    \hspace{25pt} f) $y=2 \sin(3x)$ \\
\vspace{5pt}

Daerah definisi adalah daerah nilai yang dapat menghasilkan nilai $y$.\\
Daerah fungsi adalah daerah nilai $y$ yang dapat dihasilkan dari daerah definisi.\\

a) $y=4x^2$ ~~~~~ daerah definisi:  $-\infty < x < \infty$\\
\hspace*{51pt}~~~~~~~~~ daerah fungsi:  $0 < y < \infty$\\

b) $y=2x^2+1$~~~~~ daerah definisi:  $-\infty < x < \infty$\\
\hspace*{70pt}~~~~~~~~ daerah fungsi:  $1 < y < \infty$\\

c) $y=x^2-4$~~~~~~~ daerah definisi: $-\infty < x < \infty$\\
\hspace*{70pt}~~~~~~~~ daerah fungsi:  $-4 < y < \infty$\\

d) $y=\sqrt{9-x^2} $~~~~~daerah definisi: $-3 < x < 3$\\
\hspace*{70pt}~~~~~~~~ daerah fungsi:  $0 < y < 3$\\

e) $y=^2 logx $~~~~~~~~ daerah definisi: $ x > 0$\\
\hspace*{70pt}~~~~~~~~ daerah fungsi:  $-\infty < x < \infty$\\

f) $y=2 \sin(3x)$~~~~ daerah definisi:  $-\infty < x < \infty$\\
\hspace*{70pt}~~~~~~~~ daerah fungsi:  $-2 < y < 2$\\

\section{Macam-macam Fungsi}
\subsection{Banyaknya peubah bebas}
\begin{enumerate}
\item $y=2x+1\longrightarrow$ fungsi 1 peubah.
\item $y=x^2+4x+5\longrightarrow$ fungsi 1 peubah.
\item $F(x,y)= x + 2y+5\longrightarrow$ fungsi 2 peubah.
\item $F = x^2+y^2+z^2\longrightarrow$ fungsi 3 peubah.
\end{enumerate}

\subsection{Cara Penyajian}
\subsubsection{Fungsi eksplisit}
$x$ dan $y$ ditulis dalam ruas yang terpisah.\\
a). $y = x^2 + x -5$\\
b). $y = ^3\log x$
\subsubsection{Fungsi implisit}
$x$ dan $y$ ditulis dalam ruas yang sama, dilambangkan $f(x,y)=0$.\\
a).  $x^2+y-10=0$\\
b).  $xy+2=0$
\subsubsection{Fungsi parametrik}
$x$ dan $y$ tidak terhubung secara langsung.\\
$y=2t^2+4t~~\longrightarrow$ parameter t \\
$x=10+t$
\section{Jenis Fungsi}
\subsection{Fungsi Aljabar}
$y$ disebut fungsi aljabar dari $x$ jika $y$ adalah akar persamaan derajat $n$ dalam $y$ dengan koefisien-koefisien suku-suku dalam $x$.\\

a) $y^2-4xy+x=0$ \\

b) $y^3+2x^2y^2-xy+3x+1=0$\\

$y=\dfrac{x+2}{x+1} \rightarrow$ fungsi aljabar pecahan rasional \\

$y=\sqrt{x+4} \rightarrow$ fungsi aljabar irasional \\

%\subsection{Fungsi Trigonometri}

\subsection{Fungsi Transendent}
Fungsi yang bukan fungsi aljabar\\
$y=x^n \longrightarrow n=$ bilangan irasional\\
Fungsi eksponensial: $y=a^x, ~~ a > 0, ~~ a \neq 1.$\\
Fungsi logaritmik: $y=^a log x, ~~ a>0, ~~a \neq 1$\\
Fungsi Hiperbolik: $y=\sinh {x}$\\
Fungsi Cyclometri: $y=\arcsin {x}$ \\

\subsection{Fungi Genap dan Fungsi Gasal}

Fungi $f(x)$ dikatakan sebagai fungsi genap jika $f(-x) = f(x)$\\
a) $f(x) = \cos(x)$\\
\hspace*{11pt}$f(-x) = \cos(-x) = \cos(x) \Longrightarrow f(x) = -f(x)$\\
Fungsi $f(x)$ dikatakan fungsi gasal jika $f(-x) = -f(x)$ \\
b) $f(x) = \sin{x}$\\
\hspace*{12pt}$f(-x) = \sin(-x) = -\sin(x) \Longrightarrow f(-x) = -f(x)$\\

Secara geometris, ciri fungsi genap $\rightarrow$ simetris terhadap sumbu $y$ \\
\hspace*{91px} ciri fungsi gasal $\rightarrow$ simetris terhadap titik $(0,0)$

Soal: $f(x) = x^2 + 2$

\subsection{Fungsi Periodik}
Fungsi $y$ dikatakan periodik dengan periode T jika $f(x+T)=f(x)$\\
Contoh: Berapakah periode dari:\\
a) $y=2\sin(3x)$ \\
b) $y=4 \tan (2x)$ \\
c) $y=4 \cos (6x+\frac{\pi}{3})$ \\

\subsection{Fungsi Homogen}
Fungsi $f(x)$ disebut fungsi homogen berderajat $n$ dalam peubah $x$ dan $y$ jika berlaku: \\
$$ f(\lambda x, \lambda y) = \lambda^n.f(x,y)$$
Contoh:\\
a) $f(x,y) = \sqrt[3]{x^3+y^3}$\\
b) $f(x,y) = 2xy -y^2$\\
c) $f(x,y) = xy^3-x^3y$\\

\subsection{Fungsi Naik dan Fungsi Turun}
Fungsi $f(x)$ disebut fungsi naik jika gradien fungsi tersebut $f'(x) > 0$ \\
Fungsi $f(x)$ disebut fungsi turun jika gradien fungsi tersebut $f'(x) < 0$ \\
Contoh:\\
Dapatkan intervel fungsi naik dan fungsi turun dari: \\
a) $y=x^2-2x+3$\\
b) $y=2x^3-3x^2-36x+4$\\

\subsection{Fungsi Invers}
Fungsi invers diperoleh dengan mencerminkan semua titik fungsi $f(x)$ pada garis $y=x$.\\
Contoh:\\
Dapatkan invers dari fungsi $f(x)=\dfrac{x}{2}$

\includegraphics[width=1.5in]{pict/fungsiInvers}  \\
$\Longrightarrow$ Kesimpulan untuk memperoleh fungsi invers. \\

Ambil salah satu titik pada garis $y=\dfrac{x}{2}$, misalnya $A(2,1) \Longrightarrow$ Cari persamaan garis AB. \\
Gradien garis $\overline{AB}$ adalah $m=-1$ ($\perp$ garis $y=x$)\\
Persamaan garis $\overline{AB}$: $y-1=-1(x-2) \Longrightarrow y=-x+3$ \\
Cari koordinat B (perpotongan garis $y=x$ dan $y=-x+3$) \\
$y=x$ \\
$y=-x+3  \Longrightarrow x=-x+3 \rightarrow x=\dfrac{3}{2} \rightarrow y=\dfrac{3}{2}$

$$B=(\dfrac{3}{2}, \dfrac{3}{2}). \rightarrow A= ..?$$

$x_B=\dfrac {x_A+x_A'}{2} \Longrightarrow \dfrac{3}{2} = \dfrac{3}{2} = \dfrac {2+x_A'}{2} \Longrightarrow x_A'=1$\\

$y_B=\dfrac {y_A+y_A'}{2} \Longrightarrow \dfrac{3}{2} = \dfrac{3}{2} = \dfrac {1+y_A'}{2} \Longrightarrow y_A'=1$ \\
	
$A'=(1,2)$\\
	
Persamaan garis $\overline{OA}:~~~y=\dfrac{2}{1}x \Longrightarrow y=2x$ \\

\underline{Kesimpulan:} garis $y=\dfrac{x}{2} \Longrightarrow$ dicerminkan terhadap $y=x$,\\
\hskip10pt menjadi $y=2x$ = hasil pencerminan = hasil invers. \\

Untuk ringkasnya, $y=\dfrac{x}{2}$, invers diperoleh dengan mengubah : \\$$ y \rightarrow x ~~~dan ~~~x \rightarrow y$$\\
Contoh: Dapatkan invers fungsi $y=2x$\\
\end{document}
